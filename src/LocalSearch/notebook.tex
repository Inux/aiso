
% Default to the notebook output style

    


% Inherit from the specified cell style.




    
\documentclass[11pt]{article}

    
    
    \usepackage[T1]{fontenc}
    % Nicer default font (+ math font) than Computer Modern for most use cases
    \usepackage{mathpazo}

    % Basic figure setup, for now with no caption control since it's done
    % automatically by Pandoc (which extracts ![](path) syntax from Markdown).
    \usepackage{graphicx}
    % We will generate all images so they have a width \maxwidth. This means
    % that they will get their normal width if they fit onto the page, but
    % are scaled down if they would overflow the margins.
    \makeatletter
    \def\maxwidth{\ifdim\Gin@nat@width>\linewidth\linewidth
    \else\Gin@nat@width\fi}
    \makeatother
    \let\Oldincludegraphics\includegraphics
    % Set max figure width to be 80% of text width, for now hardcoded.
    \renewcommand{\includegraphics}[1]{\Oldincludegraphics[width=.8\maxwidth]{#1}}
    % Ensure that by default, figures have no caption (until we provide a
    % proper Figure object with a Caption API and a way to capture that
    % in the conversion process - todo).
    \usepackage{caption}
    \DeclareCaptionLabelFormat{nolabel}{}
    \captionsetup{labelformat=nolabel}

    \usepackage{adjustbox} % Used to constrain images to a maximum size 
    \usepackage{xcolor} % Allow colors to be defined
    \usepackage{enumerate} % Needed for markdown enumerations to work
    \usepackage{geometry} % Used to adjust the document margins
    \usepackage{amsmath} % Equations
    \usepackage{amssymb} % Equations
    \usepackage{textcomp} % defines textquotesingle
    % Hack from http://tex.stackexchange.com/a/47451/13684:
    \AtBeginDocument{%
        \def\PYZsq{\textquotesingle}% Upright quotes in Pygmentized code
    }
    \usepackage{upquote} % Upright quotes for verbatim code
    \usepackage{eurosym} % defines \euro
    \usepackage[mathletters]{ucs} % Extended unicode (utf-8) support
    \usepackage[utf8x]{inputenc} % Allow utf-8 characters in the tex document
    \usepackage{fancyvrb} % verbatim replacement that allows latex
    \usepackage{grffile} % extends the file name processing of package graphics 
                         % to support a larger range 
    % The hyperref package gives us a pdf with properly built
    % internal navigation ('pdf bookmarks' for the table of contents,
    % internal cross-reference links, web links for URLs, etc.)
    \usepackage{hyperref}
    \usepackage{longtable} % longtable support required by pandoc >1.10
    \usepackage{booktabs}  % table support for pandoc > 1.12.2
    \usepackage[inline]{enumitem} % IRkernel/repr support (it uses the enumerate* environment)
    \usepackage[normalem]{ulem} % ulem is needed to support strikethroughs (\sout)
                                % normalem makes italics be italics, not underlines
    

    
    
    % Colors for the hyperref package
    \definecolor{urlcolor}{rgb}{0,.145,.698}
    \definecolor{linkcolor}{rgb}{.71,0.21,0.01}
    \definecolor{citecolor}{rgb}{.12,.54,.11}

    % ANSI colors
    \definecolor{ansi-black}{HTML}{3E424D}
    \definecolor{ansi-black-intense}{HTML}{282C36}
    \definecolor{ansi-red}{HTML}{E75C58}
    \definecolor{ansi-red-intense}{HTML}{B22B31}
    \definecolor{ansi-green}{HTML}{00A250}
    \definecolor{ansi-green-intense}{HTML}{007427}
    \definecolor{ansi-yellow}{HTML}{DDB62B}
    \definecolor{ansi-yellow-intense}{HTML}{B27D12}
    \definecolor{ansi-blue}{HTML}{208FFB}
    \definecolor{ansi-blue-intense}{HTML}{0065CA}
    \definecolor{ansi-magenta}{HTML}{D160C4}
    \definecolor{ansi-magenta-intense}{HTML}{A03196}
    \definecolor{ansi-cyan}{HTML}{60C6C8}
    \definecolor{ansi-cyan-intense}{HTML}{258F8F}
    \definecolor{ansi-white}{HTML}{C5C1B4}
    \definecolor{ansi-white-intense}{HTML}{A1A6B2}

    % commands and environments needed by pandoc snippets
    % extracted from the output of `pandoc -s`
    \providecommand{\tightlist}{%
      \setlength{\itemsep}{0pt}\setlength{\parskip}{0pt}}
    \DefineVerbatimEnvironment{Highlighting}{Verbatim}{commandchars=\\\{\}}
    % Add ',fontsize=\small' for more characters per line
    \newenvironment{Shaded}{}{}
    \newcommand{\KeywordTok}[1]{\textcolor[rgb]{0.00,0.44,0.13}{\textbf{{#1}}}}
    \newcommand{\DataTypeTok}[1]{\textcolor[rgb]{0.56,0.13,0.00}{{#1}}}
    \newcommand{\DecValTok}[1]{\textcolor[rgb]{0.25,0.63,0.44}{{#1}}}
    \newcommand{\BaseNTok}[1]{\textcolor[rgb]{0.25,0.63,0.44}{{#1}}}
    \newcommand{\FloatTok}[1]{\textcolor[rgb]{0.25,0.63,0.44}{{#1}}}
    \newcommand{\CharTok}[1]{\textcolor[rgb]{0.25,0.44,0.63}{{#1}}}
    \newcommand{\StringTok}[1]{\textcolor[rgb]{0.25,0.44,0.63}{{#1}}}
    \newcommand{\CommentTok}[1]{\textcolor[rgb]{0.38,0.63,0.69}{\textit{{#1}}}}
    \newcommand{\OtherTok}[1]{\textcolor[rgb]{0.00,0.44,0.13}{{#1}}}
    \newcommand{\AlertTok}[1]{\textcolor[rgb]{1.00,0.00,0.00}{\textbf{{#1}}}}
    \newcommand{\FunctionTok}[1]{\textcolor[rgb]{0.02,0.16,0.49}{{#1}}}
    \newcommand{\RegionMarkerTok}[1]{{#1}}
    \newcommand{\ErrorTok}[1]{\textcolor[rgb]{1.00,0.00,0.00}{\textbf{{#1}}}}
    \newcommand{\NormalTok}[1]{{#1}}
    
    % Additional commands for more recent versions of Pandoc
    \newcommand{\ConstantTok}[1]{\textcolor[rgb]{0.53,0.00,0.00}{{#1}}}
    \newcommand{\SpecialCharTok}[1]{\textcolor[rgb]{0.25,0.44,0.63}{{#1}}}
    \newcommand{\VerbatimStringTok}[1]{\textcolor[rgb]{0.25,0.44,0.63}{{#1}}}
    \newcommand{\SpecialStringTok}[1]{\textcolor[rgb]{0.73,0.40,0.53}{{#1}}}
    \newcommand{\ImportTok}[1]{{#1}}
    \newcommand{\DocumentationTok}[1]{\textcolor[rgb]{0.73,0.13,0.13}{\textit{{#1}}}}
    \newcommand{\AnnotationTok}[1]{\textcolor[rgb]{0.38,0.63,0.69}{\textbf{\textit{{#1}}}}}
    \newcommand{\CommentVarTok}[1]{\textcolor[rgb]{0.38,0.63,0.69}{\textbf{\textit{{#1}}}}}
    \newcommand{\VariableTok}[1]{\textcolor[rgb]{0.10,0.09,0.49}{{#1}}}
    \newcommand{\ControlFlowTok}[1]{\textcolor[rgb]{0.00,0.44,0.13}{\textbf{{#1}}}}
    \newcommand{\OperatorTok}[1]{\textcolor[rgb]{0.40,0.40,0.40}{{#1}}}
    \newcommand{\BuiltInTok}[1]{{#1}}
    \newcommand{\ExtensionTok}[1]{{#1}}
    \newcommand{\PreprocessorTok}[1]{\textcolor[rgb]{0.74,0.48,0.00}{{#1}}}
    \newcommand{\AttributeTok}[1]{\textcolor[rgb]{0.49,0.56,0.16}{{#1}}}
    \newcommand{\InformationTok}[1]{\textcolor[rgb]{0.38,0.63,0.69}{\textbf{\textit{{#1}}}}}
    \newcommand{\WarningTok}[1]{\textcolor[rgb]{0.38,0.63,0.69}{\textbf{\textit{{#1}}}}}
    
    
    % Define a nice break command that doesn't care if a line doesn't already
    % exist.
    \def\br{\hspace*{\fill} \\* }
    % Math Jax compatability definitions
    \def\gt{>}
    \def\lt{<}
    % Document parameters
    \title{03\_local\_search}
    
    
    

    % Pygments definitions
    
\makeatletter
\def\PY@reset{\let\PY@it=\relax \let\PY@bf=\relax%
    \let\PY@ul=\relax \let\PY@tc=\relax%
    \let\PY@bc=\relax \let\PY@ff=\relax}
\def\PY@tok#1{\csname PY@tok@#1\endcsname}
\def\PY@toks#1+{\ifx\relax#1\empty\else%
    \PY@tok{#1}\expandafter\PY@toks\fi}
\def\PY@do#1{\PY@bc{\PY@tc{\PY@ul{%
    \PY@it{\PY@bf{\PY@ff{#1}}}}}}}
\def\PY#1#2{\PY@reset\PY@toks#1+\relax+\PY@do{#2}}

\expandafter\def\csname PY@tok@w\endcsname{\def\PY@tc##1{\textcolor[rgb]{0.73,0.73,0.73}{##1}}}
\expandafter\def\csname PY@tok@c\endcsname{\let\PY@it=\textit\def\PY@tc##1{\textcolor[rgb]{0.25,0.50,0.50}{##1}}}
\expandafter\def\csname PY@tok@cp\endcsname{\def\PY@tc##1{\textcolor[rgb]{0.74,0.48,0.00}{##1}}}
\expandafter\def\csname PY@tok@k\endcsname{\let\PY@bf=\textbf\def\PY@tc##1{\textcolor[rgb]{0.00,0.50,0.00}{##1}}}
\expandafter\def\csname PY@tok@kp\endcsname{\def\PY@tc##1{\textcolor[rgb]{0.00,0.50,0.00}{##1}}}
\expandafter\def\csname PY@tok@kt\endcsname{\def\PY@tc##1{\textcolor[rgb]{0.69,0.00,0.25}{##1}}}
\expandafter\def\csname PY@tok@o\endcsname{\def\PY@tc##1{\textcolor[rgb]{0.40,0.40,0.40}{##1}}}
\expandafter\def\csname PY@tok@ow\endcsname{\let\PY@bf=\textbf\def\PY@tc##1{\textcolor[rgb]{0.67,0.13,1.00}{##1}}}
\expandafter\def\csname PY@tok@nb\endcsname{\def\PY@tc##1{\textcolor[rgb]{0.00,0.50,0.00}{##1}}}
\expandafter\def\csname PY@tok@nf\endcsname{\def\PY@tc##1{\textcolor[rgb]{0.00,0.00,1.00}{##1}}}
\expandafter\def\csname PY@tok@nc\endcsname{\let\PY@bf=\textbf\def\PY@tc##1{\textcolor[rgb]{0.00,0.00,1.00}{##1}}}
\expandafter\def\csname PY@tok@nn\endcsname{\let\PY@bf=\textbf\def\PY@tc##1{\textcolor[rgb]{0.00,0.00,1.00}{##1}}}
\expandafter\def\csname PY@tok@ne\endcsname{\let\PY@bf=\textbf\def\PY@tc##1{\textcolor[rgb]{0.82,0.25,0.23}{##1}}}
\expandafter\def\csname PY@tok@nv\endcsname{\def\PY@tc##1{\textcolor[rgb]{0.10,0.09,0.49}{##1}}}
\expandafter\def\csname PY@tok@no\endcsname{\def\PY@tc##1{\textcolor[rgb]{0.53,0.00,0.00}{##1}}}
\expandafter\def\csname PY@tok@nl\endcsname{\def\PY@tc##1{\textcolor[rgb]{0.63,0.63,0.00}{##1}}}
\expandafter\def\csname PY@tok@ni\endcsname{\let\PY@bf=\textbf\def\PY@tc##1{\textcolor[rgb]{0.60,0.60,0.60}{##1}}}
\expandafter\def\csname PY@tok@na\endcsname{\def\PY@tc##1{\textcolor[rgb]{0.49,0.56,0.16}{##1}}}
\expandafter\def\csname PY@tok@nt\endcsname{\let\PY@bf=\textbf\def\PY@tc##1{\textcolor[rgb]{0.00,0.50,0.00}{##1}}}
\expandafter\def\csname PY@tok@nd\endcsname{\def\PY@tc##1{\textcolor[rgb]{0.67,0.13,1.00}{##1}}}
\expandafter\def\csname PY@tok@s\endcsname{\def\PY@tc##1{\textcolor[rgb]{0.73,0.13,0.13}{##1}}}
\expandafter\def\csname PY@tok@sd\endcsname{\let\PY@it=\textit\def\PY@tc##1{\textcolor[rgb]{0.73,0.13,0.13}{##1}}}
\expandafter\def\csname PY@tok@si\endcsname{\let\PY@bf=\textbf\def\PY@tc##1{\textcolor[rgb]{0.73,0.40,0.53}{##1}}}
\expandafter\def\csname PY@tok@se\endcsname{\let\PY@bf=\textbf\def\PY@tc##1{\textcolor[rgb]{0.73,0.40,0.13}{##1}}}
\expandafter\def\csname PY@tok@sr\endcsname{\def\PY@tc##1{\textcolor[rgb]{0.73,0.40,0.53}{##1}}}
\expandafter\def\csname PY@tok@ss\endcsname{\def\PY@tc##1{\textcolor[rgb]{0.10,0.09,0.49}{##1}}}
\expandafter\def\csname PY@tok@sx\endcsname{\def\PY@tc##1{\textcolor[rgb]{0.00,0.50,0.00}{##1}}}
\expandafter\def\csname PY@tok@m\endcsname{\def\PY@tc##1{\textcolor[rgb]{0.40,0.40,0.40}{##1}}}
\expandafter\def\csname PY@tok@gh\endcsname{\let\PY@bf=\textbf\def\PY@tc##1{\textcolor[rgb]{0.00,0.00,0.50}{##1}}}
\expandafter\def\csname PY@tok@gu\endcsname{\let\PY@bf=\textbf\def\PY@tc##1{\textcolor[rgb]{0.50,0.00,0.50}{##1}}}
\expandafter\def\csname PY@tok@gd\endcsname{\def\PY@tc##1{\textcolor[rgb]{0.63,0.00,0.00}{##1}}}
\expandafter\def\csname PY@tok@gi\endcsname{\def\PY@tc##1{\textcolor[rgb]{0.00,0.63,0.00}{##1}}}
\expandafter\def\csname PY@tok@gr\endcsname{\def\PY@tc##1{\textcolor[rgb]{1.00,0.00,0.00}{##1}}}
\expandafter\def\csname PY@tok@ge\endcsname{\let\PY@it=\textit}
\expandafter\def\csname PY@tok@gs\endcsname{\let\PY@bf=\textbf}
\expandafter\def\csname PY@tok@gp\endcsname{\let\PY@bf=\textbf\def\PY@tc##1{\textcolor[rgb]{0.00,0.00,0.50}{##1}}}
\expandafter\def\csname PY@tok@go\endcsname{\def\PY@tc##1{\textcolor[rgb]{0.53,0.53,0.53}{##1}}}
\expandafter\def\csname PY@tok@gt\endcsname{\def\PY@tc##1{\textcolor[rgb]{0.00,0.27,0.87}{##1}}}
\expandafter\def\csname PY@tok@err\endcsname{\def\PY@bc##1{\setlength{\fboxsep}{0pt}\fcolorbox[rgb]{1.00,0.00,0.00}{1,1,1}{\strut ##1}}}
\expandafter\def\csname PY@tok@kc\endcsname{\let\PY@bf=\textbf\def\PY@tc##1{\textcolor[rgb]{0.00,0.50,0.00}{##1}}}
\expandafter\def\csname PY@tok@kd\endcsname{\let\PY@bf=\textbf\def\PY@tc##1{\textcolor[rgb]{0.00,0.50,0.00}{##1}}}
\expandafter\def\csname PY@tok@kn\endcsname{\let\PY@bf=\textbf\def\PY@tc##1{\textcolor[rgb]{0.00,0.50,0.00}{##1}}}
\expandafter\def\csname PY@tok@kr\endcsname{\let\PY@bf=\textbf\def\PY@tc##1{\textcolor[rgb]{0.00,0.50,0.00}{##1}}}
\expandafter\def\csname PY@tok@bp\endcsname{\def\PY@tc##1{\textcolor[rgb]{0.00,0.50,0.00}{##1}}}
\expandafter\def\csname PY@tok@fm\endcsname{\def\PY@tc##1{\textcolor[rgb]{0.00,0.00,1.00}{##1}}}
\expandafter\def\csname PY@tok@vc\endcsname{\def\PY@tc##1{\textcolor[rgb]{0.10,0.09,0.49}{##1}}}
\expandafter\def\csname PY@tok@vg\endcsname{\def\PY@tc##1{\textcolor[rgb]{0.10,0.09,0.49}{##1}}}
\expandafter\def\csname PY@tok@vi\endcsname{\def\PY@tc##1{\textcolor[rgb]{0.10,0.09,0.49}{##1}}}
\expandafter\def\csname PY@tok@vm\endcsname{\def\PY@tc##1{\textcolor[rgb]{0.10,0.09,0.49}{##1}}}
\expandafter\def\csname PY@tok@sa\endcsname{\def\PY@tc##1{\textcolor[rgb]{0.73,0.13,0.13}{##1}}}
\expandafter\def\csname PY@tok@sb\endcsname{\def\PY@tc##1{\textcolor[rgb]{0.73,0.13,0.13}{##1}}}
\expandafter\def\csname PY@tok@sc\endcsname{\def\PY@tc##1{\textcolor[rgb]{0.73,0.13,0.13}{##1}}}
\expandafter\def\csname PY@tok@dl\endcsname{\def\PY@tc##1{\textcolor[rgb]{0.73,0.13,0.13}{##1}}}
\expandafter\def\csname PY@tok@s2\endcsname{\def\PY@tc##1{\textcolor[rgb]{0.73,0.13,0.13}{##1}}}
\expandafter\def\csname PY@tok@sh\endcsname{\def\PY@tc##1{\textcolor[rgb]{0.73,0.13,0.13}{##1}}}
\expandafter\def\csname PY@tok@s1\endcsname{\def\PY@tc##1{\textcolor[rgb]{0.73,0.13,0.13}{##1}}}
\expandafter\def\csname PY@tok@mb\endcsname{\def\PY@tc##1{\textcolor[rgb]{0.40,0.40,0.40}{##1}}}
\expandafter\def\csname PY@tok@mf\endcsname{\def\PY@tc##1{\textcolor[rgb]{0.40,0.40,0.40}{##1}}}
\expandafter\def\csname PY@tok@mh\endcsname{\def\PY@tc##1{\textcolor[rgb]{0.40,0.40,0.40}{##1}}}
\expandafter\def\csname PY@tok@mi\endcsname{\def\PY@tc##1{\textcolor[rgb]{0.40,0.40,0.40}{##1}}}
\expandafter\def\csname PY@tok@il\endcsname{\def\PY@tc##1{\textcolor[rgb]{0.40,0.40,0.40}{##1}}}
\expandafter\def\csname PY@tok@mo\endcsname{\def\PY@tc##1{\textcolor[rgb]{0.40,0.40,0.40}{##1}}}
\expandafter\def\csname PY@tok@ch\endcsname{\let\PY@it=\textit\def\PY@tc##1{\textcolor[rgb]{0.25,0.50,0.50}{##1}}}
\expandafter\def\csname PY@tok@cm\endcsname{\let\PY@it=\textit\def\PY@tc##1{\textcolor[rgb]{0.25,0.50,0.50}{##1}}}
\expandafter\def\csname PY@tok@cpf\endcsname{\let\PY@it=\textit\def\PY@tc##1{\textcolor[rgb]{0.25,0.50,0.50}{##1}}}
\expandafter\def\csname PY@tok@c1\endcsname{\let\PY@it=\textit\def\PY@tc##1{\textcolor[rgb]{0.25,0.50,0.50}{##1}}}
\expandafter\def\csname PY@tok@cs\endcsname{\let\PY@it=\textit\def\PY@tc##1{\textcolor[rgb]{0.25,0.50,0.50}{##1}}}

\def\PYZbs{\char`\\}
\def\PYZus{\char`\_}
\def\PYZob{\char`\{}
\def\PYZcb{\char`\}}
\def\PYZca{\char`\^}
\def\PYZam{\char`\&}
\def\PYZlt{\char`\<}
\def\PYZgt{\char`\>}
\def\PYZsh{\char`\#}
\def\PYZpc{\char`\%}
\def\PYZdl{\char`\$}
\def\PYZhy{\char`\-}
\def\PYZsq{\char`\'}
\def\PYZdq{\char`\"}
\def\PYZti{\char`\~}
% for compatibility with earlier versions
\def\PYZat{@}
\def\PYZlb{[}
\def\PYZrb{]}
\makeatother


    % Exact colors from NB
    \definecolor{incolor}{rgb}{0.0, 0.0, 0.5}
    \definecolor{outcolor}{rgb}{0.545, 0.0, 0.0}



    
    % Prevent overflowing lines due to hard-to-break entities
    \sloppy 
    % Setup hyperref package
    \hypersetup{
      breaklinks=true,  % so long urls are correctly broken across lines
      colorlinks=true,
      urlcolor=urlcolor,
      linkcolor=linkcolor,
      citecolor=citecolor,
      }
    % Slightly bigger margins than the latex defaults
    
    \geometry{verbose,tmargin=1in,bmargin=1in,lmargin=1in,rmargin=1in}
    
    

    \begin{document}
    
    
    \maketitle
    
    

    
    \hypertarget{testat-local-search}{%
\section{Testat Local Search}\label{testat-local-search}}

\hypertarget{info}{%
\subsection{Info}\label{info}}

\begin{itemize}
\tightlist
\item
  All Questions answerd in the Document itself.
\item
  Only hill climbing done so far (but as requested in Testat)
\item
  This exercise was done in a Team (Michael Nebroj \& Steve Ineichen)
\item
  Code can be found on https://github.com/Inux/aiso
\end{itemize}

    \hypertarget{local-search-algorithms}{%
\subsection{Local Search Algorithms}\label{local-search-algorithms}}

We have seen the following local search algorithms in class:

\begin{enumerate}
\def\labelenumi{\arabic{enumi}.}
\tightlist
\item
  Hill Climbing
\item
  Genetic Algorithm
\item
  Simulated Annealing
\end{enumerate}

Two test these algorithms, we want to find the shortest path connecting
several cities. Here, we can simplify and consider the aerial distance
between two cities, so that we don't have to care about how to get from
one city to another. This problem is known as the ``Traveling Sales(man)
Problem''.

Implement the algorithms to find the shortest path connecting a list of
cities of your choice! You can use the following helper functions to
plot your path and to evaluate a path cost.

    \begin{Verbatim}[commandchars=\\\{\}]
{\color{incolor}In [{\color{incolor}204}]:} \PY{k+kn}{import} \PY{n+nn}{matplotlib}\PY{n+nn}{.}\PY{n+nn}{pyplot} \PY{k}{as} \PY{n+nn}{plt}
          
          \PY{k}{def} \PY{n+nf}{plot\PYZus{}path}\PY{p}{(}\PY{n}{path}\PY{p}{,} \PY{n}{sbb}\PY{p}{)}\PY{p}{:}
              \PY{n}{fig} \PY{o}{=} \PY{n}{plt}\PY{o}{.}\PY{n}{figure}\PY{p}{(}\PY{n}{figsize}\PY{o}{=}\PY{p}{(}\PY{l+m+mi}{10}\PY{p}{,}\PY{l+m+mi}{10}\PY{p}{)}\PY{p}{)}
              \PY{n}{last\PYZus{}city} \PY{o}{=} \PY{l+s+s2}{\PYZdq{}}\PY{l+s+s2}{\PYZdq{}}
              \PY{n}{hub\PYZus{}coordinates} \PY{o}{=} \PY{n}{sbb}\PY{o}{.}\PY{n}{get\PYZus{}hub\PYZus{}locations}\PY{p}{(}\PY{p}{)}
              \PY{n}{count} \PY{o}{=} \PY{l+m+mi}{0}
              \PY{k}{for} \PY{n}{city} \PY{o+ow}{in} \PY{n}{path}\PY{p}{:}
                  \PY{k}{if} \PY{n}{last\PYZus{}city} \PY{o}{==} \PY{l+s+s2}{\PYZdq{}}\PY{l+s+s2}{\PYZdq{}}\PY{p}{:}
                      \PY{n}{first\PYZus{}city} \PY{o}{=} \PY{n}{city}\PY{p}{;}
                  \PY{k}{if} \PY{n}{last\PYZus{}city} \PY{o}{!=} \PY{l+s+s2}{\PYZdq{}}\PY{l+s+s2}{\PYZdq{}}\PY{p}{:}
                      \PY{k}{if} \PY{n}{count} \PY{o}{==} \PY{l+m+mi}{0}\PY{p}{:}
                          \PY{n}{plt}\PY{o}{.}\PY{n}{plot}\PY{p}{(}\PY{p}{[}\PY{n}{hub\PYZus{}coordinates}\PY{p}{[}\PY{n}{city}\PY{p}{]}\PY{p}{[}\PY{l+m+mi}{0}\PY{p}{]}\PY{p}{,} \PY{n}{hub\PYZus{}coordinates}\PY{p}{[}\PY{n}{last\PYZus{}city}\PY{p}{]}\PY{p}{[}\PY{l+m+mi}{0}\PY{p}{]}\PY{p}{]}\PY{p}{,} \PY{p}{[}\PY{n}{hub\PYZus{}coordinates}\PY{p}{[}\PY{n}{city}\PY{p}{]}\PY{p}{[}\PY{l+m+mi}{1}\PY{p}{]}\PY{p}{,} \PY{n}{hub\PYZus{}coordinates}\PY{p}{[}\PY{n}{last\PYZus{}city}\PY{p}{]}\PY{p}{[}\PY{l+m+mi}{1}\PY{p}{]}\PY{p}{]}\PY{p}{,} \PY{n}{marker}\PY{o}{=}\PY{l+s+s1}{\PYZsq{}}\PY{l+s+s1}{.}\PY{l+s+s1}{\PYZsq{}}\PY{p}{,} \PY{n}{color}\PY{o}{=}\PY{l+s+s1}{\PYZsq{}}\PY{l+s+s1}{blue}\PY{l+s+s1}{\PYZsq{}}\PY{p}{)}
                      \PY{k}{else}\PY{p}{:}
                          \PY{n}{plt}\PY{o}{.}\PY{n}{plot}\PY{p}{(}\PY{p}{[}\PY{n}{hub\PYZus{}coordinates}\PY{p}{[}\PY{n}{city}\PY{p}{]}\PY{p}{[}\PY{l+m+mi}{0}\PY{p}{]}\PY{p}{,} \PY{n}{hub\PYZus{}coordinates}\PY{p}{[}\PY{n}{last\PYZus{}city}\PY{p}{]}\PY{p}{[}\PY{l+m+mi}{0}\PY{p}{]}\PY{p}{]}\PY{p}{,} \PY{p}{[}\PY{n}{hub\PYZus{}coordinates}\PY{p}{[}\PY{n}{city}\PY{p}{]}\PY{p}{[}\PY{l+m+mi}{1}\PY{p}{]}\PY{p}{,} \PY{n}{hub\PYZus{}coordinates}\PY{p}{[}\PY{n}{last\PYZus{}city}\PY{p}{]}\PY{p}{[}\PY{l+m+mi}{1}\PY{p}{]}\PY{p}{]}\PY{p}{,} \PY{n}{marker}\PY{o}{=}\PY{l+s+s1}{\PYZsq{}}\PY{l+s+s1}{.}\PY{l+s+s1}{\PYZsq{}}\PY{p}{,} \PY{n}{color}\PY{o}{=}\PY{l+s+s1}{\PYZsq{}}\PY{l+s+s1}{black}\PY{l+s+s1}{\PYZsq{}}\PY{p}{)}
                  \PY{n}{plt}\PY{o}{.}\PY{n}{text}\PY{p}{(}\PY{n}{hub\PYZus{}coordinates}\PY{p}{[}\PY{n}{city}\PY{p}{]}\PY{p}{[}\PY{l+m+mi}{0}\PY{p}{]}\PY{o}{\PYZhy{}}\PY{l+m+mf}{0.75}\PY{p}{,} \PY{n}{hub\PYZus{}coordinates}\PY{p}{[}\PY{n}{city}\PY{p}{]}\PY{p}{[}\PY{l+m+mi}{1}\PY{p}{]}\PY{o}{+}\PY{l+m+mf}{1.0}\PY{p}{,} \PY{n}{city}\PY{o}{+}\PY{l+s+s2}{\PYZdq{}}\PY{l+s+s2}{ (}\PY{l+s+s2}{\PYZdq{}}\PY{o}{+}\PY{n+nb}{str}\PY{p}{(}\PY{n}{count}\PY{p}{)}\PY{o}{+}\PY{l+s+s2}{\PYZdq{}}\PY{l+s+s2}{)}\PY{l+s+s2}{\PYZdq{}}\PY{p}{,} \PY{n}{fontsize}\PY{o}{=}\PY{l+m+mi}{9}\PY{p}{)}
                  \PY{n}{last\PYZus{}city} \PY{o}{=} \PY{n}{city}\PY{p}{;}
                  
                  \PY{n}{count} \PY{o}{=} \PY{n}{count} \PY{o}{+} \PY{l+m+mi}{1}
                          
              \PY{n}{plt}\PY{o}{.}\PY{n}{plot}\PY{p}{(}\PY{p}{[}\PY{n}{hub\PYZus{}coordinates}\PY{p}{[}\PY{n}{first\PYZus{}city}\PY{p}{]}\PY{p}{[}\PY{l+m+mi}{0}\PY{p}{]}\PY{p}{,} \PY{n}{hub\PYZus{}coordinates}\PY{p}{[}\PY{n}{last\PYZus{}city}\PY{p}{]}\PY{p}{[}\PY{l+m+mi}{0}\PY{p}{]}\PY{p}{]}\PY{p}{,} \PY{p}{[}\PY{n}{hub\PYZus{}coordinates}\PY{p}{[}\PY{n}{first\PYZus{}city}\PY{p}{]}\PY{p}{[}\PY{l+m+mi}{1}\PY{p}{]}\PY{p}{,} \PY{n}{hub\PYZus{}coordinates}\PY{p}{[}\PY{n}{last\PYZus{}city}\PY{p}{]}\PY{p}{[}\PY{l+m+mi}{1}\PY{p}{]}\PY{p}{]}\PY{p}{,} \PY{n}{marker}\PY{o}{=}\PY{l+s+s1}{\PYZsq{}}\PY{l+s+s1}{.}\PY{l+s+s1}{\PYZsq{}}\PY{p}{,} \PY{n}{color}\PY{o}{=}\PY{l+s+s1}{\PYZsq{}}\PY{l+s+s1}{green}\PY{l+s+s1}{\PYZsq{}}\PY{p}{)}
              \PY{n}{plt}\PY{o}{.}\PY{n}{axis}\PY{p}{(}\PY{l+s+s1}{\PYZsq{}}\PY{l+s+s1}{equal}\PY{l+s+s1}{\PYZsq{}}\PY{p}{)}
          
          \PY{k}{def} \PY{n+nf}{evaluate\PYZus{}path}\PY{p}{(}\PY{n}{path}\PY{p}{,} \PY{n}{sbb}\PY{p}{)}\PY{p}{:}
              \PY{n}{l} \PY{o}{=} \PY{n+nb}{list}\PY{p}{(}\PY{n}{path}\PY{o}{.}\PY{n}{copy}\PY{p}{(}\PY{p}{)}\PY{p}{)}
              \PY{n}{length} \PY{o}{=} \PY{l+m+mi}{0}
              \PY{k}{for} \PY{n}{i} \PY{o+ow}{in} \PY{n+nb}{range}\PY{p}{(}\PY{n+nb}{len}\PY{p}{(}\PY{n}{l}\PY{p}{)}\PY{o}{\PYZhy{}}\PY{l+m+mi}{1}\PY{p}{)}\PY{p}{:}
                  \PY{n}{length} \PY{o}{=} \PY{n}{length} \PY{o}{+} \PY{n}{sbb}\PY{o}{.}\PY{n}{get\PYZus{}distance\PYZus{}between}\PY{p}{(}\PY{n}{l}\PY{p}{[}\PY{n}{i}\PY{p}{]}\PY{p}{,} \PY{n}{l}\PY{p}{[}\PY{n}{i}\PY{o}{+}\PY{l+m+mi}{1}\PY{p}{]}\PY{p}{)}
          
              \PY{k}{return} \PY{n}{length}
\end{Verbatim}


    Let's define the cities we want to connect and viualize our initial
path.

    \begin{Verbatim}[commandchars=\\\{\}]
{\color{incolor}In [{\color{incolor}205}]:} \PY{k+kn}{from} \PY{n+nn}{sbb} \PY{k}{import} \PY{n}{SBB}
          
          \PY{n}{sbb} \PY{o}{=} \PY{n}{SBB}\PY{p}{(}\PY{p}{)}
          \PY{n}{sbb}\PY{o}{.}\PY{n}{importData}\PY{p}{(}\PY{l+s+s1}{\PYZsq{}}\PY{l+s+s1}{linie\PYZhy{}mit\PYZhy{}betriebspunkten.json}\PY{l+s+s1}{\PYZsq{}}\PY{p}{)}
          
          \PY{n}{path} \PY{o}{=} \PY{p}{[}\PY{l+s+s1}{\PYZsq{}}\PY{l+s+s1}{Bern}\PY{l+s+s1}{\PYZsq{}}\PY{p}{,} \PY{l+s+s1}{\PYZsq{}}\PY{l+s+s1}{Luzern}\PY{l+s+s1}{\PYZsq{}}\PY{p}{]}
          \PY{n+nb}{print}\PY{p}{(}\PY{l+s+s2}{\PYZdq{}}\PY{l+s+s2}{Bern \PYZhy{} Luzern: path cost = }\PY{l+s+s2}{\PYZdq{}} \PY{o}{+} \PY{n+nb}{str}\PY{p}{(}\PY{n}{evaluate\PYZus{}path}\PY{p}{(}\PY{n}{path}\PY{p}{,} \PY{n}{sbb}\PY{p}{)}\PY{p}{)}\PY{p}{)}
          
          \PY{n}{path} \PY{o}{=} \PY{p}{[}\PY{l+s+s1}{\PYZsq{}}\PY{l+s+s1}{Luzern}\PY{l+s+s1}{\PYZsq{}}\PY{p}{,} \PY{l+s+s1}{\PYZsq{}}\PY{l+s+s1}{Bern}\PY{l+s+s1}{\PYZsq{}}\PY{p}{]}
          \PY{n+nb}{print}\PY{p}{(}\PY{l+s+s2}{\PYZdq{}}\PY{l+s+s2}{Luzern \PYZhy{} Bern: path cost = }\PY{l+s+s2}{\PYZdq{}} \PY{o}{+} \PY{n+nb}{str}\PY{p}{(}\PY{n}{evaluate\PYZus{}path}\PY{p}{(}\PY{n}{path}\PY{p}{,} \PY{n}{sbb}\PY{p}{)}\PY{p}{)}\PY{p}{)}
          
          \PY{n}{path} \PY{o}{=} \PY{p}{[}\PY{l+s+s1}{\PYZsq{}}\PY{l+s+s1}{Luzern}\PY{l+s+s1}{\PYZsq{}}\PY{p}{,} \PY{l+s+s1}{\PYZsq{}}\PY{l+s+s1}{Bern}\PY{l+s+s1}{\PYZsq{}}\PY{p}{,} \PY{l+s+s1}{\PYZsq{}}\PY{l+s+s1}{Luzern}\PY{l+s+s1}{\PYZsq{}}\PY{p}{]}
          \PY{n+nb}{print}\PY{p}{(}\PY{l+s+s2}{\PYZdq{}}\PY{l+s+s2}{Luzern \PYZhy{} Bern \PYZhy{} Luzern: path cost = }\PY{l+s+s2}{\PYZdq{}} \PY{o}{+} \PY{n+nb}{str}\PY{p}{(}\PY{n}{evaluate\PYZus{}path}\PY{p}{(}\PY{n}{path}\PY{p}{,} \PY{n}{sbb}\PY{p}{)}\PY{p}{)}\PY{p}{)}
          
          \PY{n}{plot\PYZus{}path}\PY{p}{(}\PY{n}{path}\PY{p}{,} \PY{n}{sbb}\PY{p}{)}
\end{Verbatim}


    \begin{Verbatim}[commandchars=\\\{\}]
successfully imported 2787 hubs
successfully imported 401 train lines
Bern - Luzern: path cost = 87.69281937407082
Luzern - Bern: path cost = 87.69281937407082
Luzern - Bern - Luzern: path cost = 175.38563874814164

    \end{Verbatim}

    \begin{center}
    \adjustimage{max size={0.9\linewidth}{0.9\paperheight}}{output_4_1.png}
    \end{center}
    { \hspace*{\fill} \\}
    
    This is defenitly not the best way how to connect the cities. Let's try
our first local search algorithm.

    \hypertarget{hill-climbing}{%
\subsubsection{Hill Climbing}\label{hill-climbing}}

If we have 15 cities to connect, we have 15! different possibilities.
This is already bigger than 10\^{}12. You can easily see that the
problem becomes complex very quickly. We will have problems to
systematically explore the search space. Therefore, we will use local
search algorithms to tackle this problem.

Local search algorithms start with a solution and try to improve the
solution by considering the neighbouring states. The best neighbour will
be chosen until no better can be found.

Here, we try to minimize the distance of our path. So instead of hill
climbing, we will do the opposite. Instead of trying to find the highest
hill (maximum), we're looking at the deepest valley (minimum). But
that's not a concern, we can easily change the sign to switch from a
maximization to a minimization problem.

\emph{Hints:} - use the \texttt{evaluate\_path()} function we have
defined earlier - make sure to copy lists or sets properly:
\texttt{current\_path\ =\ path.copy()} - you can convert sets to lists
by \texttt{list(my\_set)} - a neighbouring path can be found by
switching the position of two cities

    \begin{Verbatim}[commandchars=\\\{\}]
{\color{incolor}In [{\color{incolor}206}]:} \PY{k+kn}{import} \PY{n+nn}{sys} 
          \PY{k+kn}{import} \PY{n+nn}{random}
          
          \PY{k}{def} \PY{n+nf}{hill\PYZus{}climbing\PYZus{}TSP}\PY{p}{(}\PY{n}{path}\PY{p}{,} \PY{n}{sbb}\PY{p}{)}\PY{p}{:} 
              \PY{n}{search\PYZus{}path} \PY{o}{=} \PY{n+nb}{list}\PY{p}{(}\PY{n}{path}\PY{p}{)}\PY{o}{.}\PY{n}{copy}\PY{p}{(}\PY{p}{)}
              
              \PY{n}{result} \PY{o}{=} \PY{n}{search\PYZus{}path}\PY{o}{.}\PY{n}{copy}\PY{p}{(}\PY{p}{)}
              \PY{n}{new\PYZus{}result} \PY{o}{=} \PY{n}{result}\PY{o}{.}\PY{n}{copy}\PY{p}{(}\PY{p}{)}
              
              \PY{n}{max\PYZus{}iterations} \PY{o}{=} \PY{n+nb}{len}\PY{p}{(}\PY{n}{path}\PY{p}{)}\PY{o}{*}\PY{l+m+mi}{10}
              \PY{n}{find\PYZus{}result\PYZus{}attempts} \PY{o}{=} \PY{n+nb}{len}\PY{p}{(}\PY{n}{path}\PY{p}{)}
              \PY{n}{remaining\PYZus{}attempts} \PY{o}{=} \PY{n}{find\PYZus{}result\PYZus{}attempts}
              
              \PY{n}{iteration\PYZus{}count} \PY{o}{=} \PY{l+m+mi}{0}
              
              \PY{n+nb}{print}\PY{p}{(}\PY{l+s+s2}{\PYZdq{}}\PY{l+s+s2}{Max Iterations: }\PY{l+s+si}{\PYZob{}0\PYZcb{}}\PY{l+s+s2}{\PYZdq{}}\PY{o}{.}\PY{n}{format}\PY{p}{(}\PY{n}{max\PYZus{}iterations}\PY{p}{)}\PY{p}{)}
              \PY{n+nb}{print}\PY{p}{(}\PY{l+s+s2}{\PYZdq{}}\PY{l+s+s2}{Max Attempts: }\PY{l+s+si}{\PYZob{}0\PYZcb{}}\PY{l+s+s2}{\PYZdq{}}\PY{o}{.}\PY{n}{format}\PY{p}{(}\PY{n}{find\PYZus{}result\PYZus{}attempts}\PY{p}{)}\PY{p}{)}
              \PY{n+nb}{print}\PY{p}{(}\PY{l+s+s2}{\PYZdq{}}\PY{l+s+s2}{Remaining Attempts: }\PY{l+s+si}{\PYZob{}0\PYZcb{}}\PY{l+s+s2}{\PYZdq{}}\PY{o}{.}\PY{n}{format}\PY{p}{(}\PY{n}{remaining\PYZus{}attempts}\PY{p}{)}\PY{p}{)}
              \PY{n+nb}{print}\PY{p}{(}\PY{l+s+s2}{\PYZdq{}}\PY{l+s+s2}{\PYZdq{}}\PY{p}{)}
              
              \PY{k}{for} \PY{n}{iteration} \PY{o+ow}{in} \PY{n+nb}{range}\PY{p}{(}\PY{n}{max\PYZus{}iterations}\PY{p}{)}\PY{p}{:}
                  \PY{n}{iteration\PYZus{}count} \PY{o}{=} \PY{n}{iteration\PYZus{}count} \PY{o}{+} \PY{l+m+mi}{1}
                  
                  \PY{c+c1}{\PYZsh{} https://stackoverflow.com/questions/14971181/hill\PYZhy{}climbing\PYZhy{}search\PYZhy{}algorithm\PYZhy{}applied\PYZhy{}to\PYZhy{}travelling\PYZhy{}salesman}
                  \PY{c+c1}{\PYZsh{}}
                  \PY{c+c1}{\PYZsh{} \PYZhy{} start with given path, maybe its a good one already}
                  \PY{c+c1}{\PYZsh{} \PYZhy{} len(path)\PYZhy{}2 because we dont want to change start and end position}
                  \PY{c+c1}{\PYZsh{} }
                  \PY{k}{for} \PY{n}{index} \PY{o+ow}{in} \PY{n+nb}{range}\PY{p}{(}\PY{l+m+mi}{0}\PY{p}{,} \PY{n+nb}{len}\PY{p}{(}\PY{n}{path}\PY{p}{)}\PY{o}{\PYZhy{}}\PY{l+m+mi}{2}\PY{p}{)}\PY{p}{:}
                      \PY{k}{if} \PY{n}{index} \PY{o}{\PYZgt{}} \PY{l+m+mi}{0}\PY{p}{:} \PY{c+c1}{\PYZsh{}check given path normally, without swapping}
                          \PY{n}{new\PYZus{}result}\PY{p}{[}\PY{n}{index}\PY{p}{]}\PY{p}{,} \PY{n}{new\PYZus{}result}\PY{p}{[}\PY{n}{index}\PY{o}{+}\PY{l+m+mi}{1}\PY{p}{]} \PY{o}{=} \PY{n}{new\PYZus{}result}\PY{p}{[}\PY{n}{index}\PY{o}{+}\PY{l+m+mi}{1}\PY{p}{]}\PY{p}{,} \PY{n}{new\PYZus{}result}\PY{p}{[}\PY{n}{index}\PY{p}{]}
                      
                      \PY{k}{if} \PY{n}{evaluate\PYZus{}path}\PY{p}{(}\PY{n}{new\PYZus{}result}\PY{p}{,} \PY{n}{sbb}\PY{p}{)} \PY{o}{\PYZlt{}} \PY{n}{evaluate\PYZus{}path}\PY{p}{(}\PY{n}{result}\PY{p}{,} \PY{n}{sbb}\PY{p}{)}\PY{p}{:} 
                          \PY{n+nb}{print}\PY{p}{(}\PY{l+s+s2}{\PYZdq{}}\PY{l+s+s2}{INFO: sp\PYZus{}copy: }\PY{l+s+si}{\PYZob{}0\PYZcb{}}\PY{l+s+s2}{ \PYZlt{} result: }\PY{l+s+si}{\PYZob{}1\PYZcb{}}\PY{l+s+s2}{\PYZdq{}}\PY{o}{.}\PY{n}{format}\PY{p}{(}\PY{n}{evaluate\PYZus{}path}\PY{p}{(}\PY{n}{new\PYZus{}result}\PY{p}{,} \PY{n}{sbb}\PY{p}{)}\PY{p}{,} \PY{n}{evaluate\PYZus{}path}\PY{p}{(}\PY{n}{new\PYZus{}result}\PY{p}{,} \PY{n}{sbb}\PY{p}{)}\PY{p}{)}\PY{p}{)}
                          \PY{n}{result} \PY{o}{=} \PY{n}{new\PYZus{}result}\PY{o}{.}\PY{n}{copy}\PY{p}{(}\PY{p}{)}
                          \PY{n+nb}{print}\PY{p}{(}\PY{l+s+s2}{\PYZdq{}}\PY{l+s+s2}{INFO: new result \PYZhy{}\PYZgt{} score: }\PY{l+s+si}{\PYZob{}0\PYZcb{}}\PY{l+s+s2}{, path: }\PY{l+s+si}{\PYZob{}1\PYZcb{}}\PY{l+s+s2}{\PYZdq{}}\PY{o}{.}\PY{n}{format}\PY{p}{(}\PY{n}{evaluate\PYZus{}path}\PY{p}{(}\PY{n}{result}\PY{p}{,} \PY{n}{sbb}\PY{p}{)}\PY{p}{,} \PY{n}{result}\PY{p}{)}\PY{p}{)}
                      \PY{k}{else}\PY{p}{:}
                          \PY{c+c1}{\PYZsh{}no better result}
                          \PY{n}{remaining\PYZus{}attempts} \PY{o}{=} \PY{n}{remaining\PYZus{}attempts} \PY{o}{\PYZhy{}} \PY{l+m+mi}{1}
          
                          \PY{c+c1}{\PYZsh{}randomize if shifting brings no better result}
                          \PY{k}{if} \PY{n}{remaining\PYZus{}attempts} \PY{o}{\PYZlt{}}\PY{o}{=} \PY{l+m+mi}{0}\PY{p}{:}
                              \PY{n+nb}{print}\PY{p}{(}\PY{l+s+s2}{\PYZdq{}}\PY{l+s+s2}{ERROR: }\PY{l+s+si}{\PYZob{}0\PYZcb{}}\PY{l+s+s2}{ times no better result! Randomize!}\PY{l+s+s2}{\PYZdq{}}\PY{o}{.}\PY{n}{format}\PY{p}{(}\PY{n}{find\PYZus{}result\PYZus{}attempts}\PY{o}{\PYZhy{}}\PY{n}{remaining\PYZus{}attempts}\PY{p}{)}\PY{p}{)}
                              \PY{n}{r} \PY{o}{=} \PY{n}{new\PYZus{}result}\PY{p}{[}\PY{l+m+mi}{1}\PY{p}{:}\PY{p}{(}\PY{n+nb}{len}\PY{p}{(}\PY{n}{new\PYZus{}result}\PY{p}{)}\PY{o}{\PYZhy{}}\PY{l+m+mi}{1}\PY{p}{)}\PY{p}{]}
                              \PY{n}{random}\PY{o}{.}\PY{n}{shuffle}\PY{p}{(}\PY{n}{r}\PY{p}{)}
                              \PY{n}{new\PYZus{}result}\PY{p}{[}\PY{l+m+mi}{1}\PY{p}{:}\PY{p}{(}\PY{n+nb}{len}\PY{p}{(}\PY{n}{new\PYZus{}result}\PY{p}{)}\PY{o}{\PYZhy{}}\PY{l+m+mi}{1}\PY{p}{)}\PY{p}{]} \PY{o}{=} \PY{n}{r}
          
                              \PY{c+c1}{\PYZsh{}give again same amount of attempts}
                              \PY{n}{remaining\PYZus{}attempts} \PY{o}{=} \PY{n}{find\PYZus{}result\PYZus{}attempts} 
                      
              \PY{k}{return} \PY{n}{result}\PY{p}{,} \PY{n}{iteration\PYZus{}count}
\end{Verbatim}


    \begin{Verbatim}[commandchars=\\\{\}]
{\color{incolor}In [{\color{incolor}207}]:} \PY{n}{path} \PY{o}{=} \PY{p}{[}\PY{l+s+s1}{\PYZsq{}}\PY{l+s+s1}{Bern}\PY{l+s+s1}{\PYZsq{}}\PY{p}{,} \PY{l+s+s1}{\PYZsq{}}\PY{l+s+s1}{Sargans}\PY{l+s+s1}{\PYZsq{}}\PY{p}{,} \PY{l+s+s1}{\PYZsq{}}\PY{l+s+s1}{Liestal}\PY{l+s+s1}{\PYZsq{}}\PY{p}{,} \PY{l+s+s1}{\PYZsq{}}\PY{l+s+s1}{Lugano}\PY{l+s+s1}{\PYZsq{}}\PY{p}{,} \PY{l+s+s1}{\PYZsq{}}\PY{l+s+s1}{Locarno}\PY{l+s+s1}{\PYZsq{}}\PY{p}{,} \PY{l+s+s1}{\PYZsq{}}\PY{l+s+s1}{Luzern}\PY{l+s+s1}{\PYZsq{}}\PY{p}{,} \PY{l+s+s1}{\PYZsq{}}\PY{l+s+s1}{Schaan\PYZhy{}Vaduz}\PY{l+s+s1}{\PYZsq{}}\PY{p}{,} \PY{l+s+s1}{\PYZsq{}}\PY{l+s+s1}{Zermatt}\PY{l+s+s1}{\PYZsq{}}\PY{p}{]}
          
          \PY{n}{best\PYZus{}path}\PY{p}{,} \PY{n}{iterations} \PY{o}{=} \PY{n}{hill\PYZus{}climbing\PYZus{}TSP}\PY{p}{(}\PY{n}{path}\PY{p}{,} \PY{n}{sbb}\PY{p}{)}
          
          \PY{n}{plot\PYZus{}path}\PY{p}{(}\PY{n}{best\PYZus{}path}\PY{p}{,} \PY{n}{sbb}\PY{p}{)}
          \PY{n+nb}{print}\PY{p}{(}\PY{l+s+s2}{\PYZdq{}}\PY{l+s+s2}{\PYZdq{}}\PY{p}{)}
          \PY{n+nb}{print}\PY{p}{(}\PY{l+s+s2}{\PYZdq{}}\PY{l+s+s2}{Input Path:    }\PY{l+s+s2}{\PYZdq{}} \PY{o}{+} \PY{n+nb}{str}\PY{p}{(}\PY{n}{path}\PY{p}{)}\PY{p}{)}
          \PY{n+nb}{print}\PY{p}{(}\PY{l+s+s2}{\PYZdq{}}\PY{l+s+s2}{Result Path:   }\PY{l+s+s2}{\PYZdq{}} \PY{o}{+} \PY{n+nb}{str}\PY{p}{(}\PY{n}{best\PYZus{}path}\PY{p}{)}\PY{p}{)}
          \PY{n+nb}{print}\PY{p}{(}\PY{l+s+s2}{\PYZdq{}}\PY{l+s+s2}{Input Length:  }\PY{l+s+s2}{\PYZdq{}} \PY{o}{+} \PY{n+nb}{str}\PY{p}{(}\PY{n}{evaluate\PYZus{}path}\PY{p}{(}\PY{n}{path}\PY{p}{,}\PY{n}{sbb}\PY{p}{)}\PY{p}{)}\PY{p}{)}
          \PY{n+nb}{print}\PY{p}{(}\PY{l+s+s2}{\PYZdq{}}\PY{l+s+s2}{Result Length: }\PY{l+s+s2}{\PYZdq{}} \PY{o}{+} \PY{n+nb}{str}\PY{p}{(}\PY{n}{evaluate\PYZus{}path}\PY{p}{(}\PY{n}{best\PYZus{}path}\PY{p}{,}\PY{n}{sbb}\PY{p}{)}\PY{p}{)}\PY{p}{)}
          \PY{n+nb}{print}\PY{p}{(}\PY{l+s+s2}{\PYZdq{}}\PY{l+s+s2}{Iterations:    }\PY{l+s+s2}{\PYZdq{}} \PY{o}{+} \PY{n+nb}{str}\PY{p}{(}\PY{n}{iterations}\PY{p}{)}\PY{p}{)}
\end{Verbatim}


    \begin{Verbatim}[commandchars=\\\{\}]
Max Iterations: 80
Max Attempts: 8
Remaining Attempts: 8

INFO: sp\_copy: 806.4269293088845 < result: 806.4269293088845
INFO: new result -> score: 806.4269293088845, path: ['Bern', 'Liestal', 'Sargans', 'Lugano', 'Locarno', 'Luzern', 'Schaan-Vaduz', 'Zermatt']
INFO: sp\_copy: 712.5938229521732 < result: 712.5938229521732
INFO: new result -> score: 712.5938229521732, path: ['Bern', 'Liestal', 'Lugano', 'Locarno', 'Luzern', 'Sargans', 'Schaan-Vaduz', 'Zermatt']
INFO: sp\_copy: 707.554267062716 < result: 707.554267062716
INFO: new result -> score: 707.554267062716, path: ['Bern', 'Liestal', 'Lugano', 'Locarno', 'Luzern', 'Schaan-Vaduz', 'Sargans', 'Zermatt']
ERROR: 8 times no better result! Randomize!
ERROR: 8 times no better result! Randomize!
ERROR: 8 times no better result! Randomize!
ERROR: 8 times no better result! Randomize!
ERROR: 8 times no better result! Randomize!
ERROR: 8 times no better result! Randomize!
ERROR: 8 times no better result! Randomize!
ERROR: 8 times no better result! Randomize!
ERROR: 8 times no better result! Randomize!
ERROR: 8 times no better result! Randomize!
ERROR: 8 times no better result! Randomize!
ERROR: 8 times no better result! Randomize!
ERROR: 8 times no better result! Randomize!
ERROR: 8 times no better result! Randomize!
ERROR: 8 times no better result! Randomize!
INFO: sp\_copy: 605.3501579136067 < result: 605.3501579136067
INFO: new result -> score: 605.3501579136067, path: ['Bern', 'Liestal', 'Luzern', 'Locarno', 'Sargans', 'Schaan-Vaduz', 'Lugano', 'Zermatt']
INFO: sp\_copy: 605.1713138968423 < result: 605.1713138968423
INFO: new result -> score: 605.1713138968423, path: ['Bern', 'Liestal', 'Luzern', 'Locarno', 'Schaan-Vaduz', 'Sargans', 'Lugano', 'Zermatt']
ERROR: 8 times no better result! Randomize!
ERROR: 8 times no better result! Randomize!
INFO: sp\_copy: 597.2241286025006 < result: 597.2241286025006
INFO: new result -> score: 597.2241286025006, path: ['Bern', 'Liestal', 'Luzern', 'Locarno', 'Lugano', 'Schaan-Vaduz', 'Sargans', 'Zermatt']
ERROR: 8 times no better result! Randomize!
ERROR: 8 times no better result! Randomize!
ERROR: 8 times no better result! Randomize!
ERROR: 8 times no better result! Randomize!
ERROR: 8 times no better result! Randomize!
INFO: sp\_copy: 525.0462844915054 < result: 525.0462844915054
INFO: new result -> score: 525.0462844915054, path: ['Bern', 'Liestal', 'Luzern', 'Sargans', 'Schaan-Vaduz', 'Locarno', 'Lugano', 'Zermatt']
INFO: sp\_copy: 518.1667903253285 < result: 518.1667903253285
INFO: new result -> score: 518.1667903253285, path: ['Bern', 'Liestal', 'Luzern', 'Schaan-Vaduz', 'Sargans', 'Locarno', 'Lugano', 'Zermatt']
ERROR: 8 times no better result! Randomize!
ERROR: 8 times no better result! Randomize!
ERROR: 8 times no better result! Randomize!
ERROR: 8 times no better result! Randomize!
ERROR: 8 times no better result! Randomize!
ERROR: 8 times no better result! Randomize!
ERROR: 8 times no better result! Randomize!
ERROR: 8 times no better result! Randomize!
ERROR: 8 times no better result! Randomize!
ERROR: 8 times no better result! Randomize!
ERROR: 8 times no better result! Randomize!
ERROR: 8 times no better result! Randomize!
ERROR: 8 times no better result! Randomize!
ERROR: 8 times no better result! Randomize!
ERROR: 8 times no better result! Randomize!
ERROR: 8 times no better result! Randomize!
ERROR: 8 times no better result! Randomize!
ERROR: 8 times no better result! Randomize!
ERROR: 8 times no better result! Randomize!
ERROR: 8 times no better result! Randomize!
ERROR: 8 times no better result! Randomize!
ERROR: 8 times no better result! Randomize!
ERROR: 8 times no better result! Randomize!
ERROR: 8 times no better result! Randomize!
ERROR: 8 times no better result! Randomize!
ERROR: 8 times no better result! Randomize!
ERROR: 8 times no better result! Randomize!
ERROR: 8 times no better result! Randomize!
ERROR: 8 times no better result! Randomize!
ERROR: 8 times no better result! Randomize!
ERROR: 8 times no better result! Randomize!
ERROR: 8 times no better result! Randomize!
ERROR: 8 times no better result! Randomize!
ERROR: 8 times no better result! Randomize!
ERROR: 8 times no better result! Randomize!
ERROR: 8 times no better result! Randomize!
ERROR: 8 times no better result! Randomize!

Input Path:    ['Bern', 'Sargans', 'Liestal', 'Lugano', 'Locarno', 'Luzern', 'Schaan-Vaduz', 'Zermatt']
Result Path:   ['Bern', 'Liestal', 'Luzern', 'Schaan-Vaduz', 'Sargans', 'Locarno', 'Lugano', 'Zermatt']
Input Length:  1022.4946103114211
Result Length: 518.1667903253285
Iterations:    80

    \end{Verbatim}

    \begin{center}
    \adjustimage{max size={0.9\linewidth}{0.9\paperheight}}{output_8_1.png}
    \end{center}
    { \hspace*{\fill} \\}
    
    Oh, what happend here? Is this the best we can get? * (Answer Q1) With a
simple implementation we do not find the optimal Solution. Therefore a
randomization was implemented when we end up in a local minimum. (Start
and end position are not randomized). It is far better with this
optimization but it is still not a optimal solution.

\begin{itemize}
\tightlist
\item
  Why is this so?

  \begin{itemize}
  \tightlist
  \item
    (Answer Q2) We end up in a local minimum
  \end{itemize}
\item
  How many steps did we need to get to this solution?

  \begin{itemize}
  \tightlist
  \item
    When Start and End position should stay then range(0, len(path)-2)
    -\textgreater{} max swap steps to try (with initial path)
  \end{itemize}
\item
  Can you suggest a method to improve the hill climbing algorithm?

  \begin{itemize}
  \tightlist
  \item
    (Answer Q3) Randomization (see code example)
  \end{itemize}
\end{itemize}

    \hypertarget{genetic-algorithm}{%
\subsubsection{Genetic Algorithm}\label{genetic-algorithm}}

Genetic algorithms (or GA) are inspired by natural evolution and are
particularly useful in optimization and search problems with large state
spaces.

Given a problem, algorithms in the domain make use of a
\emph{population} of solutions (also called \emph{states}), where each
solution/state represents a feasible solution. At each iteration (often
called \emph{generation}), the population gets updated using methods
inspired by biology and evolution, like \emph{crossover},
\emph{mutation} and \emph{natural selection}.

A genetic algorithm works in the following way:

\begin{enumerate}
\def\labelenumi{\arabic{enumi})}
\item
  Initialize random population.
\item
  Calculate population fitness.
\item
  Select individuals for mating.
\item
  Mate selected individuals to produce new population.

  \begin{itemize}
  \tightlist
  \item
    Random chance to mutate individuals.
  \end{itemize}
\item
  Repeat from step 2) until an individual is fit enough or the maximum
  number of iterations was reached.
\end{enumerate}

Below, you can find some helper functions to implement your genetic
algorithm.

First, create a dictionnary that maps a letter to a city name.

    Our solution will be a path through all the cities. To simplify, we will
encode each city with a letter from the alphabet. So your first initial
path through the cities will have the code ``ABCDEFGHIJK..''. We can
easily convert a letter to a city by
\texttt{letter2city(\textquotesingle{}A\textquotesingle{})} or
\texttt{city2letter(\textquotesingle{}Rotkreuz\textquotesingle{})}.

    \begin{Verbatim}[commandchars=\\\{\}]
{\color{incolor}In [{\color{incolor}208}]:} \PY{k+kn}{import} \PY{n+nn}{string}
          
          \PY{n}{number\PYZus{}of\PYZus{}cities} \PY{o}{=} \PY{n+nb}{len}\PY{p}{(}\PY{n}{path}\PY{p}{)}
          \PY{n}{letter2city} \PY{o}{=} \PY{n+nb}{dict}\PY{p}{(}\PY{p}{)}
          \PY{n}{city2letter} \PY{o}{=} \PY{n+nb}{dict}\PY{p}{(}\PY{p}{)}
          \PY{k}{for} \PY{n}{i} \PY{o+ow}{in} \PY{n+nb}{range}\PY{p}{(}\PY{n}{number\PYZus{}of\PYZus{}cities}\PY{p}{)}\PY{p}{:}
              \PY{n}{letter2city}\PY{p}{[}\PY{n}{string}\PY{o}{.}\PY{n}{ascii\PYZus{}uppercase}\PY{p}{[}\PY{n}{i}\PY{p}{]}\PY{p}{]} \PY{o}{=} \PY{n+nb}{list}\PY{p}{(}\PY{n}{path}\PY{p}{)}\PY{p}{[}\PY{n}{i}\PY{p}{]}
              \PY{n}{city2letter}\PY{p}{[}\PY{n+nb}{list}\PY{p}{(}\PY{n}{path}\PY{p}{)}\PY{p}{[}\PY{n}{i}\PY{p}{]}\PY{p}{]} \PY{o}{=} \PY{n}{string}\PY{o}{.}\PY{n}{ascii\PYZus{}uppercase}\PY{p}{[}\PY{n}{i}\PY{p}{]}
              
          \PY{k}{def} \PY{n+nf}{path2string}\PY{p}{(}\PY{n}{path}\PY{p}{)}\PY{p}{:}
              \PY{n}{s} \PY{o}{=} \PY{l+s+s2}{\PYZdq{}}\PY{l+s+s2}{\PYZdq{}}
              \PY{k}{for} \PY{n}{city} \PY{o+ow}{in} \PY{n}{path}\PY{p}{:}
                  \PY{n}{s}\PY{o}{+}\PY{o}{=}\PY{n}{city2letter}\PY{p}{[}\PY{n}{city}\PY{p}{]}
              \PY{k}{return} \PY{n+nb}{list}\PY{p}{(}\PY{n}{s}\PY{p}{)}
          
          \PY{k}{def} \PY{n+nf}{path2cities}\PY{p}{(}\PY{n}{path}\PY{p}{)}\PY{p}{:}
              \PY{n}{s} \PY{o}{=} \PY{n+nb}{list}\PY{p}{(}\PY{p}{)}
              \PY{k}{for} \PY{n}{letter} \PY{o+ow}{in} \PY{n}{path}\PY{p}{:}
                  \PY{n}{s}\PY{o}{.}\PY{n}{append}\PY{p}{(}\PY{n}{letter2city}\PY{p}{[}\PY{n}{letter}\PY{p}{]}\PY{p}{)}
              \PY{k}{return} \PY{n}{s}
          
          \PY{n}{path\PYZus{}code} \PY{o}{=} \PY{n}{path2string}\PY{p}{(}\PY{n}{path}\PY{p}{)}
          \PY{n+nb}{print}\PY{p}{(}\PY{l+s+s2}{\PYZdq{}}\PY{l+s+s2}{the path has the following code : }\PY{l+s+s2}{\PYZdq{}}\PY{p}{)}
          \PY{n+nb}{print}\PY{p}{(}\PY{n}{path\PYZus{}code}\PY{p}{)}
\end{Verbatim}


    \begin{Verbatim}[commandchars=\\\{\}]
the path has the following code : 
['A', 'B', 'C', 'D', 'E', 'F', 'G', 'H']

    \end{Verbatim}

    Let's inizialize a random population:

    \begin{Verbatim}[commandchars=\\\{\}]
{\color{incolor}In [{\color{incolor}209}]:} \PY{k+kn}{import} \PY{n+nn}{random}
          
          \PY{k}{def} \PY{n+nf}{init\PYZus{}population}\PY{p}{(}\PY{n}{pop\PYZus{}number}\PY{p}{,} \PY{n}{cities}\PY{p}{)}\PY{p}{:}
              \PY{l+s+sd}{\PYZdq{}\PYZdq{}\PYZdq{}Initializes population for genetic algorithm}
          \PY{l+s+sd}{    pop\PYZus{}number  :  Number of individuals in population}
          \PY{l+s+sd}{    cities      :  cities in letter code \PYZdq{}\PYZdq{}\PYZdq{}}
\end{Verbatim}


    We can calculate the fitness of a path using the evaluate\_path
function. Note that shorter paths are considered fitter.

    \begin{Verbatim}[commandchars=\\\{\}]
{\color{incolor}In [{\color{incolor}210}]:} \PY{k}{def} \PY{n+nf}{fitness}\PY{p}{(}\PY{n}{sample}\PY{p}{)}\PY{p}{:}
\end{Verbatim}


    \begin{Verbatim}[commandchars=\\\{\}]

          File "<ipython-input-210-3263e2f4dfc5>", line 1
        def fitness(sample):
                            \^{}
    SyntaxError: unexpected EOF while parsing


    \end{Verbatim}

    Create a function to select two individuals for mating. Fitter
individuals are more likely to be selected for reproduction than less
fit individuals. Therefore, we have to calculate the weights of each
indiviudal that corresponds to the likelyhood of being chosen for
reproduction.

    \begin{Verbatim}[commandchars=\\\{\}]
{\color{incolor}In [{\color{incolor}211}]:} \PY{k+kn}{import} \PY{n+nn}{random}
          \PY{k+kn}{from} \PY{n+nn}{random} \PY{k}{import} \PY{n}{choices}
          
          
          \PY{k}{def} \PY{n+nf}{calculate\PYZus{}weights}\PY{p}{(}\PY{n}{population}\PY{p}{)}\PY{p}{:}
              \PY{c+c1}{\PYZsh{} calculate the weight of each individual}
              \PY{k}{return}
          
          
          \PY{k}{def} \PY{n+nf}{select}\PY{p}{(}\PY{n}{population}\PY{p}{,} \PY{n}{weights}\PY{p}{)}\PY{p}{:}
              \PY{c+c1}{\PYZsh{} return two individuals for reproduction}
              \PY{c+c1}{\PYZsh{} fitter individuals should be more likely to be selected}
              \PY{c+c1}{\PYZsh{} hint: use random.choices to chose from the population based on the weights of each individual}
              \PY{k}{return}
          
          \PY{n}{population} \PY{o}{=} \PY{n}{init\PYZus{}population}\PY{p}{(}\PY{l+m+mi}{10}\PY{p}{,} \PY{n}{path\PYZus{}code}\PY{p}{)}
          \PY{n}{weights} \PY{o}{=} \PY{n}{calculate\PYZus{}weights}\PY{p}{(}\PY{n}{population}\PY{p}{)}
          \PY{n+nb}{print}\PY{p}{(}\PY{n}{select}\PY{p}{(}\PY{n}{population}\PY{p}{,} \PY{n}{weights}\PY{p}{)}\PY{p}{)}
\end{Verbatim}


    \begin{Verbatim}[commandchars=\\\{\}]
None

    \end{Verbatim}

    Now that we can select two individuals, we make them reproduce using
crossover and mutation. We need to consider that we want to visit every
city exactly once. For example, for the crossover, you can take a random
lenght of individual 1 and fill up the remaining cities based on the
order of the unvisited cities in individual 2.

    \begin{Verbatim}[commandchars=\\\{\}]
{\color{incolor}In [{\color{incolor}212}]:} \PY{k}{def} \PY{n+nf}{crossover}\PY{p}{(}\PY{n}{x}\PY{p}{,} \PY{n}{y}\PY{p}{)}\PY{p}{:}
              \PY{c+c1}{\PYZsh{} create an offspring from the parents x and y}
              \PY{k}{return}
              
          \PY{k}{def} \PY{n+nf}{mutate}\PY{p}{(}\PY{n}{x}\PY{p}{,} \PY{n}{p\PYZus{}mutate}\PY{p}{)}\PY{p}{:}
              \PY{c+c1}{\PYZsh{} switch the location of two cities}
              \PY{k}{return}
          
          \PY{c+c1}{\PYZsh{} test your code}
          \PY{n}{x} \PY{o}{=} \PY{n}{path\PYZus{}code}
          \PY{n}{y} \PY{o}{=} \PY{n}{random}\PY{o}{.}\PY{n}{sample}\PY{p}{(}\PY{n}{path\PYZus{}code}\PY{p}{,} \PY{n+nb}{len}\PY{p}{(}\PY{n}{path\PYZus{}code}\PY{p}{)}\PY{p}{)}
          \PY{n}{xy} \PY{o}{=} \PY{n}{crossover}\PY{p}{(}\PY{n}{x}\PY{p}{,}\PY{n}{y}\PY{p}{)}
          \PY{n+nb}{print}\PY{p}{(}\PY{n}{x}\PY{p}{)}
          \PY{n+nb}{print}\PY{p}{(}\PY{n}{y}\PY{p}{)}
          \PY{n+nb}{print}\PY{p}{(}\PY{n}{xy}\PY{p}{)}
          \PY{n}{mutate}\PY{p}{(}\PY{n}{xy}\PY{p}{,} \PY{l+m+mf}{0.5}\PY{p}{)}
          \PY{n+nb}{print}\PY{p}{(}\PY{n}{xy}\PY{p}{)}
\end{Verbatim}


    \begin{Verbatim}[commandchars=\\\{\}]
['A', 'B', 'C', 'D', 'E', 'F', 'G', 'H']
['A', 'G', 'B', 'D', 'C', 'E', 'H', 'F']
None
None

    \end{Verbatim}

    We have now all the ingredients to create our genetic algorithm:

    \hypertarget{simulated-annealing}{%
\subsubsection{Simulated Annealing}\label{simulated-annealing}}

The intuition behind Hill Climbing was developed from the metaphor of
climbing up the graph of a function to find its peak. There is a
fundamental problem in the implementation of the algorithm however. To
find the highest hill, we take one step at a time, always uphill, hoping
to find the highest point, but if we are unlucky to start from the
shoulder of the second-highest hill, there is no way we can find the
highest one. The algorithm will always converge to the local optimum.
Hill Climbing is also bad at dealing with functions that flatline in
certain regions. If all neighboring states have the same value, we
cannot find the global optimum using this algorithm. Let's now look at
an algorithm that can deal with these situations. Simulated Annealing is
quite similar to Hill Climbing, but instead of picking the \emph{best}
move every iteration, it picks a \emph{random} move. If this random move
brings us closer to the global optimum, it will be accepted, but if it
doesn't, the algorithm may accept or reject the move based on a
probability dictated by the \emph{temperature}. When the
\emph{temperature} is high, the algorithm is more likely to accept a
random move even if it is bad. At low temperatures, only good moves are
accepted, with the occasional exception. This allows exploration of the
state space and prevents the algorithm from getting stuck at the local
optimum.

The temperature is gradually decreased over the course of the iteration.
This is done by a scheduling routine:

    \begin{Verbatim}[commandchars=\\\{\}]
{\color{incolor}In [{\color{incolor}213}]:} \PY{k}{def} \PY{n+nf}{exp\PYZus{}schedule}\PY{p}{(}\PY{n}{k}\PY{o}{=}\PY{l+m+mi}{20}\PY{p}{,} \PY{n}{lam}\PY{o}{=}\PY{l+m+mf}{0.005}\PY{p}{,} \PY{n}{limit}\PY{o}{=}\PY{l+m+mi}{100}\PY{p}{)}\PY{p}{:}
              \PY{l+s+sd}{\PYZdq{}\PYZdq{}\PYZdq{}One possible schedule function for simulated annealing\PYZdq{}\PYZdq{}\PYZdq{}}
              \PY{k}{return} \PY{k}{lambda} \PY{n}{t}\PY{p}{:} \PY{p}{(}\PY{n}{k} \PY{o}{*} \PY{n}{math}\PY{o}{.}\PY{n}{exp}\PY{p}{(}\PY{o}{\PYZhy{}}\PY{n}{lam} \PY{o}{*} \PY{n}{t}\PY{p}{)} \PY{k}{if} \PY{n}{t} \PY{o}{\PYZlt{}} \PY{n}{limit} \PY{k}{else} \PY{l+m+mi}{0}\PY{p}{)}
\end{Verbatim}



    % Add a bibliography block to the postdoc
    
    
    
    \end{document}
